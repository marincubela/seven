\chapter{Specifikacija programske potpore}

\section{Funkcionalni zahtjevi}			

\noindent \textbf{Dionici:}

\begin{packed_enum}
	
	\item Ovlašteni zaposlenik tvrtke
	\item Klijent
	\item Administrator
	\item Razvojni tim
	
\end{packed_enum}

\noindent \textbf{Aktori i njihovi funkcionalni zahtjevi:}


\begin{packed_enum}
	\item  \underbar{Neregistrirani/neprijavljeni korisnik (inicijator) može:}
	
	\begin{packed_enum}
		
		\item vidjeti kartu s parkiralištima (adresa, broj ukupnih i slobodnih mjesta, cijena)
		\item registrirati se i/ili prijaviti se (kao klijent ili tvrtka)
		
	\end{packed_enum}
	
	\item  \underbar{Klijent (inicijator) može:}
	
	\begin{packed_enum}
		
		\item vidjeti kartu s parkiralištima (adresa, broj ukupnih i slobodnih mjesta, cijena)
		\item pregledavati, uređivati i izbrisati korisnički račun
		\item pregledavati, uređivati i izbrisati vozila
		\item pregledavati, uređivati i izbrisati kartice
		\item rezervirati parkirališno mjesto
		\item pregledavati, uređivati i izbrisati rezervacije
		\item pisati i pregledavati  recenzije
		\item tražiti upute do parkirališta
		
	\end{packed_enum}
	
	
	\item  \underbar{Ovlašteni zaposlenik tvrtke (inicijator) može:}
	
	\begin{packed_enum}
		
		\item vidjeti kartu s parkiralištima
		\item pregledavati, uređivati, dodavati i brisati svoja parkirališta
		\item pregledavati, uređivati i izbrisati korisnički račun
		\item pregledavati vlastite i tuđe recenzije
		\item pregledavati rezervacije na vlastitim parkiralištima
		\item odgovoriti na recenzije korisnika
		
	\end{packed_enum}
	
	\item \underbar{Administrator (inicijator) može:}
	
	\begin{packed_enum}
		
		\item vidjeti popis svih registriranih korisnika i osnovnih podataka
		\item pregledavati rezervacije
		\item brisati i mijenjati razinu pristupa aplikaciji drugim korisnicima
		\item brisati recenzije
		\item dodavati ili brisati parkirališta
		
	\end{packed_enum}
	
	\item \underbar{Baza podataka (sudionik) može:}
	
	\begin{packed_enum}
		
		\item pohranjivati podatke o korisnicima (osobni podatci, kartica, vozila)
		\item pohranjivati podatke o parkiralištima (njihovim kapacitetima i ponudama)
		\item pohranjivati podatke o rezervacijama
		\item pohranjivati podatke o tvrtkama
		
	\end{packed_enum}
\end{packed_enum}


\eject 



\subsection{Obrasci uporabe}

\subsubsection{Opis obrazaca uporabe}

\noindent \underbar{\textbf{UC$<$broj obrasca$>$ -$<$ime obrasca$>$}}
\begin{packed_item}
	
	\item \textbf{Glavni sudionik: }$<$sudionik$>$
	\item  \textbf{Cilj:} $<$cilj$>$
	\item  \textbf{Sudionici:} $<$sudionici$>$
	\item  \textbf{Preduvjet:} $<$preduvjet$>$
	\item  \textbf{Opis osnovnog tijeka:}
	
	\item[] \begin{packed_enum}
		
		\item $<$opis korak jedan$>$
		\item $<$opis korak dva$>$
		\item $<$opis korak tri$>$
		\item $<$opis korak četiri$>$
		\item $<$opis korak pet$>$
	\end{packed_enum}
	
	\item  \textbf{Opis mogućih odstupanja:}
	
	\item[] \begin{packed_item}
		
		\item[2.a] $<$opis mogućeg scenarija odstupanja u koraku 2$>$
		\item[] \begin{packed_enum}
			
			\item $<$opis rješenja mogućeg scenarija korak 1$>$
			\item $<$opis rješenja mogućeg scenarija korak 2$>$
			
		\end{packed_enum}
		\item[2.b] $<$opis mogućeg scenarija odstupanja u koraku 2$>$
		\item[3.a] $<$opis mogućeg scenarija odstupanja  u koraku 3$>$
		
	\end{packed_item}
\end{packed_item}

\noindent \underbar{\textbf{UC$<$broj obrasca$>$ -$<$ime obrasca$>$}}
\begin{packed_item}
	
	\item \textbf{Glavni sudionik: }$<$sudionik$>$
	\item  \textbf{Cilj:} $<$cilj$>$
	\item  \textbf{Sudionici:} $<$sudionici$>$
	\item  \textbf{Preduvjet:} $<$preduvjet$>$
	\item  \textbf{Opis osnovnog tijeka:}
	
	\item[] \begin{packed_enum}
		
		\item $<$opis korak jedan$>$
		\item $<$opis korak dva$>$
		\item $<$opis korak tri$>$
		\item $<$opis korak četiri$>$
		\item $<$opis korak pet$>$
	\end{packed_enum}
	
	\item  \textbf{Opis mogućih odstupanja:}
	
	\item[] \begin{packed_item}
		
		\item[2.a] $<$opis mogućeg scenarija odstupanja u koraku 2$>$
		\item[] \begin{packed_enum}
			
			\item $<$opis rješenja mogućeg scenarija korak 1$>$
			\item $<$opis rješenja mogućeg scenarija korak 2$>$
			
		\end{packed_enum}
		\item[2.b] $<$opis mogućeg scenarija odstupanja u koraku 2$>$
		\item[3.a] $<$opis mogućeg scenarija odstupanja  u koraku 3$>$
		
	\end{packed_item}
\end{packed_item}

\subsubsection{Dijagrami obrazaca uporabe}

\textit{Prikazati odnos aktora i obrazaca uporabe odgovarajućim UML dijagramom. Nije nužno nacrtati sve na jednom dijagramu. Modelirati po razinama apstrakcije i skupovima srodnih funkcionalnosti.}
\eject		

\subsection{Sekvencijski dijagrami}

\textbf{\textit{dio 1. revizije}}\\

\textit{Nacrtati sekvencijske dijagrame koji modeliraju najvažnije dijelove sustava (max. 4 dijagrama). Ukoliko postoji nedoumica oko odabira, razjasniti s asistentom. Uz svaki dijagram napisati detaljni opis dijagrama.}
\eject

\section{Ostali zahtjevi}

\textbf{\textit{dio 1. revizije}}\\

\textit{Nefunkcionalni zahtjevi i zahtjevi domene primjene dopunjuju funkcionalne zahtjeve. Oni opisuju \textbf{kako se sustav treba ponašati} i koja \textbf{ograničenja} treba poštivati (performanse, korisničko iskustvo, pouzdanost, standardi kvalitete, sigurnost...). Primjeri takvih zahtjeva u Vašem projektu mogu biti: podržani jezici korisničkog sučelja, vrijeme odziva, najveći mogući podržani broj korisnika, podržane web/mobilne platforme, razina zaštite (protokoli komunikacije, kriptiranje...)... Svaki takav zahtjev potrebno je navesti u jednoj ili dvije rečenice.}



