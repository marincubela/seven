\chapter{Specifikacija programske potpore}

\section{Funkcionalni zahtjevi}			

\noindent \textbf{Dionici:}

\begin{packed_enum}
	
	\item Ovlašteni zaposlenik tvrtke
	\item Klijent
	\item Administrator
	\item Razvojni tim
	
\end{packed_enum}

\noindent \textbf{Aktori i njihovi funkcionalni zahtjevi:}


\begin{packed_enum}
	\item  \underbar{Neregistrirani/neprijavljeni korisnik (inicijator) može:}
	
	\begin{packed_enum}
		
		\item vidjeti kartu s parkiralištima (adresa, broj ukupnih i slobodnih mjesta, cijena)
		\item registrirati se i/ili prijaviti se (kao klijent ili tvrtka)
		
	\end{packed_enum}
	
	\item  \underbar{Klijent (inicijator) može:}
	
	\begin{packed_enum}
		
		\item vidjeti kartu s parkiralištima (adresa, broj ukupnih i slobodnih mjesta, cijena)
		\item pregledavati, uređivati i izbrisati korisnički račun
		\item pregledavati, uređivati i izbrisati vozila
		\item pregledavati, uređivati i izbrisati kartice
		\item rezervirati parkirališno mjesto
		\item pregledavati, uređivati i izbrisati rezervacije
		\item pisati i pregledavati  recenzije
		\item tražiti upute do parkirališta
		
	\end{packed_enum}
	
	
	\item  \underbar{Ovlašteni zaposlenik tvrtke (inicijator) može:}
	
	\begin{packed_enum}
		
		\item vidjeti kartu s parkiralištima
		\item pregledavati, uređivati, dodavati i brisati svoja parkirališta
		\item pregledavati, uređivati i izbrisati korisnički račun
		\item pregledavati vlastite i tuđe recenzije
		\item pregledavati rezervacije na vlastitim parkiralištima
		\item odgovoriti na recenzije korisnika
		
	\end{packed_enum}
	
	\item \underbar{Administrator (inicijator) može:}
	
	\begin{packed_enum}
		
		\item vidjeti popis svih registriranih korisnika i osnovnih podataka
		\item pregledavati rezervacije
		\item brisati i mijenjati razinu pristupa aplikaciji drugim korisnicima
		\item brisati recenzije
		\item dodavati ili brisati parkirališta
		
	\end{packed_enum}
	
	\item \underbar{Baza podataka (sudionik) može:}
	
	\begin{packed_enum}
		
		\item pohranjivati podatke o korisnicima (osobni podatci, kartica, vozila)
		\item pohranjivati podatke o parkiralištima (njihovim kapacitetima i ponudama)
		\item pohranjivati podatke o rezervacijama
		\item pohranjivati podatke o tvrtkama
		
	\end{packed_enum}
\end{packed_enum}


\eject 



\subsection{Obrasci uporabe}

\subsubsection{Opis obrazaca uporabe}

\noindent \underbar{\textbf{UC1 - Pristup privatnim rutama}}
\begin{packed_item}
	
	\item \textbf{Glavni sudionik:} Korisnik
	\item \textbf{Cilj:} Zabraniti pristup privatnim rutama
	\item \textbf{Sudionici:} -
	\item \textbf{Preduvjet:} Korisnik nije prijavljen
	\item \textbf{Opis osnovnog tijeka:}
	
	\item[] \begin{packed_enum}
		
		\item Pristup privatnoj ruti se zabranjuje te se korisnik preusmjerava na formu za prijavu
		\item Korisnik se prijavljuje te se preusmjerava na prethodnu zabranjenu rutu

	\end{packed_enum}
	
	\item  \textbf{Opis mogućih odstupanja:}
	
	\item[] \begin{packed_item}
		
		\item[2.a] Korisnik pokušava pristupiti rutama koje njemu nisu dostupne
		\item[] \begin{packed_enum}
			
			\item Nakon prijave korisnik se preusmjerava na početni zaslon aplikacije
			
		\end{packed_enum}
		
	\end{packed_item}
\end{packed_item}

\noindent \underbar{\textbf{UC2 - Pregled registriranih korisnika}}
\begin{packed_item}
	
	\item \textbf{Glavni sudionik:} Administrator
	\item  \textbf{Cilj:} Pregled podataka o registriranim korisnicima
	\item  \textbf{Sudionici:} Baza podataka
	\item  \textbf{Preduvjet:} Administrator je prijavljen
	\item  \textbf{Opis osnovnog tijeka:}
	
	\item[] \begin{packed_enum}
		
		\item Administrator na admin panelu odabire opciju pregledavanja svih registriranih korisnika 
		\item Prikaže se popis svih korisnika koji su registrirani u aplikaciju i njihovi osnovni podatci (ime, prezime, email, vrijeme zadnje prijave, je li račun potvrđen, broj registriranih vozila)

	\end{packed_enum}
\end{packed_item}

\noindent \underbar{\textbf{UC3 - Pregled registriranog korisnika}}
\begin{packed_item}
	
	\item \textbf{Glavni sudionik:} Administrator
	\item  \textbf{Cilj:} Pregledati podatke o registriranom korisniku
	\item  \textbf{Sudionici:} Baza podataka
	\item  \textbf{Preduvjet:} Administrator je prijavljen
	\item  \textbf{Opis osnovnog tijeka:}
	
	\item[] \begin{packed_enum}

		\item Administrator na admin panelu odabire opciju pregledavanja svih registriranih korisnika 
		\item Na popisu registriranih korisnika odabire jednog korisnika
		\item Prikaže se popis svih podataka o odabranom korisniku (ime, prezime, email adresa, vrijeme zadnje prijave, je li račun potvrđen, popis registriranih vozila, postoji li način plačanja)

	\end{packed_enum}
\end{packed_item}

\noindent \underbar{\textbf{UC4 - Uklanjanje registriranog korisnika}}
\begin{packed_item}
	
	\item \textbf{Glavni sudionik:} Administrator
	\item  \textbf{Cilj:} Ukloniti registriranog korisnika
	\item  \textbf{Sudionici:} Baza podataka
	\item  \textbf{Preduvjet:} Administrator je prijavljen
	\item  \textbf{Opis osnovnog tijeka:}
	
	\item[] \begin{packed_enum}
		
		\item Administrator na admin panelu odabire opciju pregledavanja svih registriranih korisnika 
		\item Prikaže se popis svih korisnika koji su registrirani u aplikaciju te odabire korisnika kojeg želi izbrisati
		\item Administrator odabire akciju ”Ukoni korisnika i sve njegove podatke"
		\item Korisnik se više ne može ulogirati te je potrebna nova registracija

	\end{packed_enum}
\end{packed_item}

\noindent \underbar{\textbf{UC5 - Uklanjanje parkirališta}}
\begin{packed_item}
	
	\item \textbf{Glavni sudionik:} Administrator, Tvrtka
	\item  \textbf{Cilj:} Ukloniti parkiralište
	\item  \textbf{Sudionici:} Baza podataka
	\item  \textbf{Preduvjet:} Administrator/Tvrtka je prijavljen
	\item  \textbf{Opis osnovnog tijeka:}
	
	\item[] \begin{packed_enum}
		
		\item Administrator/Tvrtka otvara popis parkirališta te se odabire parkiralište koje se želi ukloniti
		\item Prikažu se detalji parkirališta
		\item Administrator/Tvrtka odabire akciju ”Ukoni parkiralište"
		\item Parkiralište više nije vidljivo na karti niti na popisu parkirališta

	\end{packed_enum}
\end{packed_item}

\noindent \underbar{\textbf{UC6 - Dodavanje parkirališta}}
\begin{packed_item}

	\item \textbf{Glavni sudionik:} Administrator, Tvrtka
	\item  \textbf{Cilj:} Dodati parkiralište
	\item  \textbf{Sudionici:} Baza podataka
	\item  \textbf{Preduvjet:} Administrator/Tvrtka je prijavljen
	\item  \textbf{Opis osnovnog tijeka:}
	
	\item[] \begin{packed_enum}
		
		\item Administrator/Tvrtka otvara formu za dodavanje parkirališta
		\item Unose se svi potrebni podaci
		\item Administrator/Tvrtka odabire akciju ”Dodaj parkiralište"
		\item Parkiralište postaje javno vidljivo na karti i u popisu parkirališta svim korisnicima aplikacije

	\end{packed_enum}
\end{packed_item}

\noindent \underbar{\textbf{UC7 - Registracija korisnika}}
\begin{packed_item}

	\item \textbf{Glavni sudionik:} Korisnik
	\item  \textbf{Cilj:} Registrirati korisnika
	\item  \textbf{Sudionici:} Baza podataka
	\item  \textbf{Preduvjet:} Korisnik nije prijavljen niti registriran
	\item  \textbf{Opis osnovnog tijeka:}
	
	\item[] \begin{packed_enum}
		
		\item Korisnik odabire opciju za registraciju korisnika
		\item Otvara se forma za popunjavanje korisničkih podataka
		\item Korisnik otvara novi račun i prikazuje mu se poruka da je račun uspješno stvoren
		\item Korisnik je automatski ulogiran u sustav i otvara mu se čarobnjak za popunjavanje preostalih podataka o računu
		\item Kada je čarobnjak završen preusmjeren je na početni zaslon aplikacije

	\end{packed_enum}

	\item  \textbf{Opis mogućih odstupanja:}
	
	\item[] \begin{packed_item}
	
		\item[3.a] Odabir već postojećeg emaila
		\item[] \begin{packed_enum}
			
			\item Korisnik se obavještava o neuspjeloj registraciji
			\item Korisnik koristi druge podatke te pokušava ponovno ili odustaje od registracije
			
		\end{packed_enum}
		
		\item[4.a] Odabir već postojećeg emaila
		\item[] \begin{packed_enum}
			
			\item Prilikom prve sljedeće uspješne prijave, čarobnjak se opet prikazuje te se nastavlja na koraku koji nije završen
			
		\end{packed_enum}
		
	\end{packed_item}
\end{packed_item}

\noindent \underbar{\textbf{UC8 - Prijava korisnika}}
\begin{packed_item}

	\item \textbf{Glavni sudionik:} Korisnik
	\item  \textbf{Cilj:} Prijaviti korisnika
	\item  \textbf{Sudionici:} Baza podataka
	\item  \textbf{Preduvjet:} Korisnik nije prijavljen
	\item  \textbf{Opis osnovnog tijeka:}
	
	\item[] \begin{packed_enum}
		
		\item Korisnik odabire opciju za prijavu korisnika
		\item Otvara se forma za popunjavanje korisničkih podataka potrebnih za prijavu
		\item Korisnik je prijavljen u aplikaciju i preusmjeren na početni zaslon aplikacije

	\end{packed_enum}

	\item  \textbf{Opis mogućih odstupanja:}
	
	\item[] \begin{packed_item}
	
		\item[2.a] Pogrešni podaci prilikom prijave
		\item[] \begin{packed_enum}
			
			\item Prikazuje se poruka o pogrešnim podacima za prijavu 
			\item Korisnik pokušava ponovno unijeti ispravne podatke za prijavu ili odustaje od prijave
			
		\end{packed_enum}
		
		\item[2.b] Korisnik je zaboravio lozinku za prijavu
		\item[] \begin{packed_enum}
			
			\item Prikazuje se poruka o pogrešnim podacima za prijavu 
			\item Korisnik odabire akciju "Zaboravio sam lozinku"
			\item Otvara se forma za unos emaila
			\item Na email adresu mu se šalje poveznica koja ga vodi na formu za postavljanje nove lozinke te ju korisnik popunjava
			\item Lozinka je promijenjena
			\item Korisnik se preusmjerava na formu za prijavu
			
		\end{packed_enum}

	\end{packed_item}
\end{packed_item}

\subsubsection{Dijagrami obrazaca uporabe}

\textit{Prikazati odnos aktora i obrazaca uporabe odgovarajućim UML dijagramom. Nije nužno nacrtati sve na jednom dijagramu. Modelirati po razinama apstrakcije i skupovima srodnih funkcionalnosti.}
\eject		

\subsection{Sekvencijski dijagrami}

\textbf{\textit{dio 1. revizije}}\\

\textit{Nacrtati sekvencijske dijagrame koji modeliraju najvažnije dijelove sustava (max. 4 dijagrama). Ukoliko postoji nedoumica oko odabira, razjasniti s asistentom. Uz svaki dijagram napisati detaljni opis dijagrama.}
\eject

\section{Ostali zahtjevi}

\textbf{\textit{dio 1. revizije}}\\

\textit{Nefunkcionalni zahtjevi i zahtjevi domene primjene dopunjuju funkcionalne zahtjeve. Oni opisuju \textbf{kako se sustav treba ponašati} i koja \textbf{ograničenja} treba poštivati (performanse, korisničko iskustvo, pouzdanost, standardi kvalitete, sigurnost...). Primjeri takvih zahtjeva u Vašem projektu mogu biti: podržani jezici korisničkog sučelja, vrijeme odziva, najveći mogući podržani broj korisnika, podržane web/mobilne platforme, razina zaštite (protokoli komunikacije, kriptiranje...)... Svaki takav zahtjev potrebno je navesti u jednoj ili dvije rečenice.}