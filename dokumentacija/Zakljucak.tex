\chapter{Zaključak i budući rad}
		
		Naš je zadatak bio napraviti web aplikaciju koja će korisnicima omogućiti pretraživanje i rezervaciju slobodnih parkirališnih mjesta u gradu Zagrebu svih tvrtki koje odluče nuditi parkiralište preko naše aplikacije. 
		
		Rad na aplikaciji podijelili smo u dvije faze: planiranje i izrada dokumentacije i generičkih funkcionalnosti aplikacije te implementacija projekta, ispitivanje i dovršetak dokumentacije.
		
		Prvu fazu započeli smo okupljanjem razvojnog tima. Nakon što smo dobili projektni zadatak, počeli smo s dokumentiranjem i dogovaranjem kako će naša aplikacija izgledati. Tu smo naišli i na prve izazove. Nitko od nas nije radio sličan projekt i nije znao što očekivati niti kako započeti rad na takvom projektu. Mislili smo da trebamo što prije početi programirati, ali čim smo započeli shvatili smo kako prvo trebamo dobro definirati svojstva naše aplikacije, sve zahtjeve i moguće scenarije pa smo to i napravili. Izrađena dokumentacija (obrasci uporabe, dijagram baze podataka itd.) pomogla nam je da se bolje upoznamo sa zadatkom, a kasnije nudila jasne smjernice kako bi naša aplikacija trebala izgledati.
		
		Podijelili smo se u tri grupe koje su radile na određenim dijelovima aplikacije: frontend, backend i baze podataka. Grupa zadužena za bazu podataka se nakon izrade baze pridružila \textit{backendu} i \textit{frontendu}. Prvu smo fazu završili implementacijom registracije i prijave za korisnike.
		
		Druga faza većinom se sastojala od implementacije aplikacije. Jasno definirani zahtjevi napisani u dokumentaciji ubrzali su rad na aplikaciji. Svatko je znao što treba raditi i kako bi to trebalo izgledati na kraju. Na kraju druge faze dovršili smo ispitivanje aplikacije i preostalu dokumentaciju koja će poslužiti za buduće inačice sustava.
		
		Radeći na ovom projektu morali smo riješiti mnoge izazove. Koristili smo potpuno nove tehnologije koje smo prvo morali naučiti da bi ih dobro upotrijebili. Implementacija karte, postavljanje aplikacije na javno dostupnom poslužitelju, komunikacija baze podataka, \textit{backend} i \textit{frontend} dijela samo su neki od izazova koje smo uspješno savladali. Naučili smo raditi u grupi, rješavati konflikte, koristiti nove tehnologije i ne bojati se probati nešto novo. 
		
		Neki izazovi su bili tehnički prezahtjevni za nas da ih riješimo u okviru ovog zadatka. Javni web poslužitelj kojeg smo koristili ne omogućuje nam da dohvaćamo trenutnu lokaciju korisnika i spremamo kolačiće na njegovu domenu. Spremanje kolačića smo riješili kupnjom vlastite domene, dok dohvaćanje trenutne lokacije sigurno ostaje kao daljnja mogućnost poboljšanja aplikacije. Uz to, poboljšanje ove aplikacije može se ostvariti izgradnjom mobilne aplikacije što je jedan od budućih proširenja sustava.
		
		Ovaj projektni zadatak bio je vrijedno iskustvo svim članovima grupe. Iskusili smo kako je biti dio tima, raditi stvari u zadanom roku. Naučili smo koliko dobra organizacija i priprema pomažu i ubrzavaju rad na projektu, ali i koliko još imamo za učiti i da će uvijek biti mjesta za napredak.
		
		\eject 