\chapter{Implementacija i korisničko sučelje}

		
		\section{Korištene tehnologije i alati}
		
			Komunikacija u timu realizirana je korištenjem aplikacije \underline{WhatsApp} \footnote{\href{https://www.whatsapp.com/}{Whatsapp}} za manje i kraće dogovore, a za dulje i veće dogovore komunikacija je realizirana korištenjem aplikacije \underline{Microsoft Teams} \footnote{\href{https://www.microsoft.com/hr-hr/microsoft-365/microsoft-teams/group-chat-software}{Microsoft Teams}}. Za izradu UML dijagrama korišten je alat \underline{Astah UML} \footnote{\href{https://astah.net/products/astah-uml/}{Astah UML}}. Kao sustav za upravljanje izvornim kodom Git \footnote{\href{https://git-scm.com/}{Git}}. Udaljeni repozitorij projekta je dostupan na web platformi \underline{GitLab} \footnote{\href{https://www.gitlab.com}{Gitlab}}.

			Kao razvojno okruženje korišten je \underline{Microsoft Visual Studio Code} \footnote{\href{https://code.visualstudio.com/}{Microsoft Visual Studio Code}}. Microsoft Visual Studio Code trenutno je \underline{najpopularnije} \footnote{\href{https://pypl.github.io/IDE.html}{Deset najraširenijih razvojnih okruženja}} razvojno okruženje, iznimno je prilagodljiv raznim potrebama programera i potrebama samog programskog jezika.
			
			Aplikacija je izrađena u dva dijela. Backend dio je napisan koristeći radni okvir \underline{Express.js} \footnote{\href{https://expressjs.com/}{Express.js}} u programskom jeziku \underline{TypeScript} \footnote{\href{https://www.typescriptlang.org/}{TypeScript}}, a frontend dio je napisan koristeći biblioteku \underline{React.js} \footnote{\href{https://reactjs.org/}{React.js}} u programskom jeziku \underline{JavaScript} \footnote{\href{https://www.javascript.com/}{JavaScript}}.
			
			Express.js radni okvir je javno dostupan, besplatan te otvorenog izvora pružen od strane \underline{OpenJS Foundation-a} \footnote{\href{https://openjsf.org/}{OpenJS Foundation}}. Express.js je jednostavna i prilagodljiv \underline{Node.js}\footnote{\href{https://nodejs.org/en/}{Node.js}} radni okvir koja pruža izdržljiv skup značajki za web i mobilne aplikacije.
			
			React.js je JavaScript biblioteka za izradu interaktivnih sučelja unutar web aplikacija. Održavana je od strane Facebooka, a najčešće je korišten kao osnova u razvoju web aplikacija.
			
			Korištena baza podataka je \underline{PostgreSQL} \footnote{\href{https://www.postgresql.org/}{PostgreSQL}}. PostgreSQL je besplatna, objektno-relacijska baza podataka sa više od 30 godina aktivnog razvoja.
			
			Cijela aplikacija poslužena je na \underline{Heroku} \footnote{\href{https://www.heroku.com/home}{Heroku}}. Heroku je platforma koja omogućuje jednostavno i sigurno puštanje  u pogon i nadziranje aplikacija. Konkretno, koristimo dvije Heroku aplikacije, jednu za backend i jednu za frontend te u sklopu backend aplikacije koristimo i jednu instancu PostrgeSQL baze podataka.
			
			\eject 
		
	
		\section{Ispitivanje programskog rješenja}
			
			\textbf{\textit{dio 2. revizije}}\\
			
			 \textit{U ovom poglavlju je potrebno opisati provedbu ispitivanja implementiranih funkcionalnosti na razini komponenti i na razini cijelog sustava s prikazom odabranih ispitnih slučajeva. Studenti trebaju ispitati temeljnu funkcionalnost i rubne uvjete.}
	
			
			\subsection{Ispitivanje komponenti}
			\textit{Potrebno je provesti ispitivanje jedinica (engl. unit testing) nad razredima koji implementiraju temeljne funkcionalnosti. Razraditi \textbf{minimalno 6 ispitnih slučajeva} u kojima će se ispitati redovni slučajevi, rubni uvjeti te izazivanje pogreške (engl. exception throwing). Poželjno je stvoriti i ispitni slučaj koji koristi funkcionalnosti koje nisu implementirane. Potrebno je priložiti izvorni kôd svih ispitnih slučajeva te prikaz rezultata izvođenja ispita u razvojnom okruženju (prolaz/pad ispita). }
			
			
			
			\subsection{Ispitivanje sustava}
			
			 \textit{Potrebno je provesti i opisati ispitivanje sustava koristeći radni okvir Selenium\footnote{\url{https://www.seleniumhq.org/}}. Razraditi \textbf{minimalno 4 ispitna slučaja} u kojima će se ispitati redovni slučajevi, rubni uvjeti te poziv funkcionalnosti koja nije implementirana/izaziva pogrešku kako bi se vidjelo na koji način sustav reagira kada nešto nije u potpunosti ostvareno. Ispitni slučaj se treba sastojati od ulaza (npr. korisničko ime i lozinka), očekivanog izlaza ili rezultata, koraka ispitivanja i dobivenog izlaza ili rezultata.\\ }
			 
			 \textit{Izradu ispitnih slučajeva pomoću radnog okvira Selenium moguće je provesti pomoću jednog od sljedeća dva alata:}
			 \begin{itemize}
			 	\item \textit{dodatak za preglednik \textbf{Selenium IDE} - snimanje korisnikovih akcija radi automatskog ponavljanja ispita	}
			 	\item \textit{\textbf{Selenium WebDriver} - podrška za pisanje ispita u jezicima Java, C\#, PHP koristeći posebno programsko sučelje.}
			 \end{itemize}
		 	\textit{Detalji o korištenju alata Selenium bit će prikazani na posebnom predavanju tijekom semestra.}
			
			\eject 
		
		
		\section{Dijagram razmještaja}
			
			\textbf{\textit{dio 2. revizije}}
			
			 \textit{Potrebno je umetnuti \textbf{specifikacijski} dijagram razmještaja i opisati ga. Moguće je umjesto specifikacijskog dijagrama razmještaja umetnuti dijagram razmještaja instanci, pod uvjetom da taj dijagram bolje opisuje neki važniji dio sustava.}
			
			\eject 
		
		\section{Upute za puštanje u pogon}
				
			\subsection{Frontend aplikacija}
			
			
			
			\subsection{Backend aplikacija}
		
		
			\subsection{Baza podataka}
		
		    \subsubsection*{Lokalna instalacija}
		    
		        Potrebno je preuzeti pgAdmin aplikaciju uz koju standardno dolazi i PostgreSQL server. Pokrenuti instalaciju te uz pgAdmin klijenta odabrati i PostgreSQL server za instalaciju. Nakon instalacije potrebno je postaviti korisnika. Nakon instalacije pgAdmin klijent i sam PostgreSQL server su pokrenuti.
		        
		    \subsubsection*{Lokalna konfiguracija}
		        
		        Nakon instalacije, potrebno je napraviti novu lokalnu bazu.
		        
		        \begin{enumerate}
		            \item Proširiti \textit{Servers} te, ovisno o preuzetoj verziji PostgreSQL servera, proširiti  \textit{PostgreSQL XX}
		            \item Desni klik na \textit{Databases} te \textit{Create} te \textit{Database...}
		            \item Otvara se modal te kao \textit{Database} vrijednost stavljamo proizvoljni naziv
		            \item Odabiremo \textit{Spremi} te time stvaramo novu bazu
		        \end{enumerate}
		        
		        Nakon stvorene baze potrebno je stvoriti konekcijski \textit{connection string}. Njega ćemo stvoriti što ćemo vrijednosti u vitičastim zagradama zamijeniti s pravim vrijednostima.
		        
		        \begin{lstlisting}
postgres://{username}:{password}@{hostname}:{port}/{database}
		        \end{lstlisting}
		        
		        \begin{enumerate}
		            \item username - inicijalni username je \textit{postgres}
		            \item password - lozinka postavljena prilikom instalacije
		            \item hostname - pošto je server lokalni, hostname će biti \textit{localhost}
		            \item port - inicijalni priključak PostgreSQL servera je \textit{5432}
		            \item database - ime novo-napravljene baze
		        \end{enumerate}
		        
		        Nakon generiranog \textit{connection stringa}, kreirat ćemo \textit{.env} datoteku u korijenskom direktoriju backend aplikacije te u njoj napraviti zapis:
		        
		        \begin{verbatim}
		            DATABASE_URL={connectionString}
		        \end{verbatim}
		        
		        Sada će prilikom pokretanja backend aplikacije, prije samog spajanja s bazom, aplikacija pročitati sadržaj \textit{.env} datoteke i učitati te vrijednosti kao sistemske varijable. U Node.js aplikaciji tim varijablama imamo pristup preko
		        
		        \begin{verbatim}
		            proccess.env.\${imeVarijable}
		        \end{verbatim}
		
		    \subsubsection*{Heroku instalacija}
		    
		         \begin{enumerate}
		            \item Otići na Heroku Dashboard
		            \item Odabrati kreiranu backend aplikaciju
		            \item Odabrati karticu \textit{Resources}
		            \item Unutar \textit{Add-ons} tražilice tražiti \textbf{Heroku Postgres}
		            \item Kada se otvori modal, odabrati \textbf{Submit Order Form}
		        \end{enumerate}
		        
		        I to je to! Server sada ima svoju bazu. Ukoliko se želimo spojiti na tu bazu preko pgAdmin klijenta tada slijedimo sljedeće korake:
		        
		         \begin{enumerate}
		            \item Sa \textit{Resources} zaslona odaberemo naš \textit{Heroku Postgres} server
		            \item Nakon malo duljeg učitavanja, odabiremo karticu \textit{Settings}
		            \item Odabiremo gumb \textit{View Credentials...}
		        \end{enumerate}
		        
		        Sada na su nam na ekranu prikazani svi potrebni podaci za spajanje na PostgreSQL server. Sljedeći korak je otići u pgAdmin klijent te kreirati novu konekciju.
		        
		         \begin{enumerate}
		            \item Otvoriti pgAdmin
		            \item Desnim klikom kliknuti na \textit{Servers} te odabrati \textit{Create} \textit{Server...}
		            \item Postaviti proizvoljno ime (ne utječe na samu konekciju)
		            \item Odabrati karticu \textit{Connection} te tu popuniti potrebne podatke (\textit{Hostname/address}, \textit{Maintenece database}, \textit{Username}, \textit{Password})
		            \item Nakon popunjenih podataka potrebno je kliknuti na gumb \textit{Save}
		        \end{enumerate}
		        
		        Sada smo stvorili novu konekciju te iz popisa servera možemo odabrati novo-kreirani te manipulirati s tablicama.
		        
		
		    \subsubsection*{Heroku konfiguracija}
		        
		        Slično kao u lokalnoj konfiguraciji potrebno je igrati se s varijablamo. Jedina i vrlo bitna razlika je ta da sve već posloženo te na Heroku serveru već postoji varijabla \textit{DATABASE\_URL}. Potrebno je samo pustiti backend aplikaciju u pogon.
		    