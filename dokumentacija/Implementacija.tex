\chapter{Implementacija i korisničko sučelje}
		
		\section{Korištene tehnologije i alati}
		
			Komunikacija u timu realizirana je korištenjem aplikacije \underline{WhatsApp} \footnote{\href{https://www.whatsapp.com/}{Whatsapp}} za manje i
			kraće dogovore, a za dulje i veće dogovore komunikacija je realizirana korištenjem aplikacije \underline{Microsoft Teams} \footnote{\href{https://www.microsoft.com/hr-hr/microsoft-365/microsoft-teams/group-chat-software}{Microsoft Teams}}. Za izradu UML dijagrama korišten je alat \underline{Astah UML} \footnote{\href{https://astah.net/products/astah-uml/}{Astah UML}}. Kao sustav za upravljanje izvornim kodom Git \footnote{\href{https://git-scm.com/}{Git}}. Udaljeni repozitorij projekta je dostupan na web platformi \underline{GitLab} \footnote{\href{https://www.gitlab.com}{Gitlab}}.

            Kao razvojno okruženje korišten je \underline{Microsoft Visual Studio Code} \footnote{\href{https://code.visualstudio.com/}{Microsoft Visual Studio Code}}. Microsoft Visual Studio Code trenutno je \underline{najpopularnije} \footnote{\href{https://pypl.github.io/IDE.html}{Deset najraširenijih razvojnih okruženja}} razvojno okruženje, iznimno je prilagodljiv raznim potrebama programera i potrebama samog programskog jezika.
            
            Aplikacija je izrađena u dva dijela. Backend dio je napisan koristeći radni okvir \underline{Express.js} \footnote{\href{https://expressjs.com/}{Express.js}} u programskom jeziku \underline{TypeScript} \footnote{\href{https://www.typescriptlang.org/}{TypeScript}}, a frontend dio je napisan koristeći biblioteku \underline{React.js} \footnote{\href{https://reactjs.org/}{React.js}} u programskom jeziku \underline{JavaScript} \footnote{\href{https://www.javascript.com/}{JavaScript}}.
            
            Express.js radni okvir je javno dostupan, besplatan te otvorenog izvora pružen od strane \underline{OpenJS Foundation-a} \footnote{\href{https://openjsf.org/}{OpenJS Foundation}}. Express.js je jednostavna i prilagodljiv \underline{Node.js}\footnote{\href{https://nodejs.org/en/}{Node.js}} radni okvir koja pruža izdržljiv skup značajki za web i mobilne aplikacije.
            
            React.js je JavaScript biblioteka za izradu interaktivnih sučelja unutar web aplikacija. Održavana je od strane Facebooka, a najčešće je korišten kao osnova u razvoju web aplikacija.
            
            Korištena baza podataka je \underline{PostgreSQL} \footnote{\href{https://www.postgresql.org/}{PostgreSQL}}. PostgreSQL je besplatna, objektno-relacijska baza podataka sa više od 30 godina aktivnog razvoja.
            
            Cijela aplikacija poslužena je na \underline{Heroku} \footnote{\href{https://www.heroku.com/home}{Heroku}}. Heroku je platforma koja omogućuje jednostavno i sigurno puštanje  u pogon i nadziranje aplikacija. Konkretno, koristimo dvije Heroku aplikacije, jednu za backend i jednu za frontend te u sklopu backend aplikacije koristimo i jednu instancu PostrgeSQL baze podataka.
			
			\eject 
		
	
		\section{Ispitivanje programskog rješenja}
			
			Ispitivanje aplikacije proveli smo pomoću testnih radnih okvira Jest\footnote{\href{https://jestjs.io/}{Jest}} za ispitivanje jedinica i Selenium\footnote{www.selenium.dev} za ispitivanje sustava.
	
			
			\subsection{Ispitivanje komponenti}
			\textit{Potrebno je provesti ispitivanje jedinica (engl. unit testing) nad razredima koji implementiraju temeljne funkcionalnosti. Razraditi \textbf{minimalno 6 ispitnih slučajeva} u kojima će se ispitati redovni slučajevi, rubni uvjeti te izazivanje pogreške (engl. exception throwing). Poželjno je stvoriti i ispitni slučaj koji koristi funkcionalnosti koje nisu implementirane. Potrebno je priložiti izvorni kôd svih ispitnih slučajeva te prikaz rezultata izvođenja ispita u razvojnom okruženju (prolaz/pad ispita). }
			
			Ispitivanje komponenti provedeno je pomoću Jest testnog radnog okvira. Ispitivali smo metode na \textit{backend} dijelu aplikacije.
			
			
			
			\subsection{Ispitivanje sustava}
			
			 \textit{Potrebno je provesti i opisati ispitivanje sustava koristeći radni okvir Selenium\footnote{\url{https://www.seleniumhq.org/}}. Razraditi \textbf{minimalno 4 ispitna slučaja} u kojima će se ispitati redovni slučajevi, rubni uvjeti te poziv funkcionalnosti koja nije implementirana/izaziva pogrešku kako bi se vidjelo na koji način sustav reagira kada nešto nije u potpunosti ostvareno. Ispitni slučaj se treba sastojati od ulaza (npr. korisničko ime i lozinka), očekivanog izlaza ili rezultata, koraka ispitivanja i dobivenog izlaza ili rezultata.\\ }
			
			Ispitivanjem sustava želimo provjeriti osnovne funkcionalnosti aplikacije koje korisnik susreće.
			
			Prvi test ispituje prijavu korisnika na aplikaciju. Nakon uspješne prijave korisnik je preusmjeren na početnu stranicu i prijavljen je.
			
			
			
			Drugi test provjerava neispravnu registraciju tvrtke kad korisnik unese pogrešan OIB. Na zaslon se ispiše poruka i korisnik ostaje na istoj stranici i omogućava mu se ponovni unos podataka. Ako je registracija uspješna, korisnik se preusmjeri na početnu stranicu aplikacije.
			
			
			Treći test provjerava dodavanje vozila. Korisnik se prijavi, odabere stranicu 'Moja vozila', klikne na opciju 'Dodaj', popuni podatke i pošalje zahtjev. Nakon što je vozilo uspješno dodano, korisnika se preusmjerava na stranicu s njegovim vozilima.
			
			\eject 
		
		
		\section{Dijagram razmještaja}
			
			
			 Dijagram razmještaja koristi se za prikaz topologije sustava pri čemu je naglasak stavljen na odnose između sklopovskih i programskih komponenti. Oni opisuju korištenu programsku potporu sustava te njegove elemente i okolinu izvođenja. Osnovni su elementi čvorovi, artefakti te spojevi koji predstavljaju komunikacijske puteve između njih.
			 
			 Klijent putem svojeg web preglednika pristupa web aplikaciji koja se vrti na cloud serveru. Sva komunikacija odvija se putem protokola HTTP. Na poslužiteljskom računalu nalaze se baza podataka te aplikacija.
			 \begin{figure}[H]
    			\includegraphics[width=1\linewidth]{dijagrami/Deployment Diagram.png}
    			\caption{Dijagram razmještaja}
    			\label{fig:Dijagram razmještaja} 
    		\end{figure}
			\eject 
		
		\section{Upute za puštanje u pogon}
		
		    Prije ikakve radnje, potrebno je klonirati \textit{GitLab} repozitorij sa kodom te otvoriti terminal (i opcionalno sistemski preglednik) u tom direktoriju. Udaljeni \textit{GitLab} repozitorij možete pronaći na  \underline{\href{https://gitlab.com/Cubi5/seven}{ovoj poveznici}}
		    
		    Kako bi kloniranje bilo moguće potrebno je imati instalirani \textit{Git} klijent na računalu. \textit{Git} klijent moguće je preuzeti na \underline{\href{https://git-scm.com/downloads}{ovoj poveznici}}.
		    
		    Još jedan uvjet kako bi aplikacija radila lokalno jest instaliran \textit{Node.js}. \textit{Node.js} moguće je preuzeti na \underline{\href{https://nodejs.org/en/download/}{ovoj poveznici}}. Instalacijom \textit{Node.js} aplikacije, instalirat će vam se i \textit{\underline{npm}} \footnote{\href{https://www.npmjs.com/}{npm}} koji je zapravo abrevijacija od \textit{Node Package Manager}. Samo ime govori da je to alat za upravljanje \textit{Node.js} paketima.

		        
				\pagebreak
		

				\subsection{Frontend aplikacija}
	        
						\subsubsection*{Lokalna instalacija}
	        
	          		Da bismo pokrenuli lokalnu frontend aplikaciju potrebno je unutar terminala pozicionirati se u frontend mapu unutar korijenskog direktorija. 
	            
								\begin{center}
										\texttt{cd frontend}
								\end{center}
								
								Sada je potrebno instalirati potrebne pakete za uspješan rad aplikacije. To činimo pokretanjem naredbe
								
								\begin{center}
										\texttt{npm install}
								\end{center}
								
								Nakon instalacije paketa, potrebno je podesiti i lokalni \textit{host} file. Da bismo uredili \textit{host} file prvo ga je potrebno locirati, a sama lokacija ovisi o operacijskom sustavu. Ukoliko koristimo operacijski sustav na bazi \textit{Linuxa} \textit{host} file se nalazi na lokaciji \texttt{/etc/hosts}, dok se na \textit{Windows} operacijskom sustavu ta datoteka nalazi na \verb$C:\Windows\System32\drivers\etc\hosts$
								
								Jednom kada smo je locirali, potrebno je na kraj datoteke dodati redak

								\begin{center}
										\texttt{127.0.0.1	frontend-local.parkirajme.xyz}
								\end{center}

								Također, potrebno je unutar \textit{frontend} direktrorija kreirati datoteku \textit{.env} te u nju dodati sljedeći zapis

								\begin{center}
										\texttt{HOST=frontend-local.parkirajme.xyz}
								\end{center}
								
								Dodavanjem ovih zapisa, završili smo lokalnu instalaciju servera te server možemo pokrenuti naredbom
								
								\begin{center}
										\texttt{npm run dev}
								\end{center}
								
								Kako bi sve ispravno radilo lokalnom serveru ćemo pristupiti sa sljedeće adrese

								\begin{center}
										\href{http://frontend-local.parkirajme.xyz:3000/}{\texttt{frontend-local.parkirajme.xyz:3000}}
								\end{center}
		             
						\pagebreak
		    
						\subsubsection*{Heroku instalacija}
	        
								Kako bismo frontend aplikaciju pustili u pogon, potrebne je kreirati novu aplikaciji na Heroku platformi.
								
								\begin{enumerate}
									\item Otvoriti \href{https://dashboard.heroku.com/apps}{heroku.com} stranicu i prijaviti se ukoliko to već nismo prethodno učinili.
									\item U gornjem desnom kutu možemo vidjeti gumb \textit{New} te kada ga pritisnemo otvara nam se padajuća lista i u njoj odabiremo \textit{Create new app}
									\item Otvara nam se forma za kreiranje nove aplikacije te tu navodimo ime aplikacije koje može biti proizvoljno. Također na tom koraku odabiremo i regiju servera i tu je poželjno postaviti server koji je nama najbliži, a to je Europa.
									\item Pritiskom na \textit{Create app} stvaramo aplikaciju
		        		\end{enumerate}
		        
		        		Sada kada je Heroku aplikacija napravljena potrebno je otvoriti \texttt{.git/config} datoteku unutar korijenskog direktorija aplikacije te na kraj te datoteke potrebno je dodati

                \begin{verbatim}
    [remote "heroku-frontend"]
        	url = https://git.heroku.com/${imeHerokuAplikacije}.git
        	fetch = +refs/heads/*:refs/remotes/heroku-frontend/*
								\end{verbatim}
								
								te umjesto \texttt{\{imeHerokuAplikacije\}} potrebno je umetnuti ime novostvorene aplikacije. Kada smo podesili \texttt{config} datoteku sve što je potrebno napraviti je pozicionirati se u direktorij frontend aplikacije te iz terminala pokrenuti 

								\begin{center}
										\texttt{npm run deploy}
								\end{center}
								
								Nakon par minuta aplikacija će biti puštena u pogon.

						\pagebreak

						\subsubsection*{HTTPS lokalna instalacija}

								Kako bi lokalno pokrenuli aplikaciju sa HTTPS protokolom potrebno je u datoteku \textit{frontend/.env} dodati sljedeći zapis

								\begin{center}
										\texttt{HTTPS=true}
								\end{center}

								Sada, prilikom sljedećeg pokretanja lokalnog servera, aplikacija će se pokrenuti na HTTPS protokolu te će browser moći pročitati našu trenutnu lokaciju.

						\pagebreak

						\subsection{Backend aplikacija}
	        
						\subsubsection*{Lokalna instalacija}
	        
	          		Da bismo pokrenulni lokalnu backend aplikaciju potrebno je unutar terminala pozicionirati se u backend mapu unutar korijenskog direktorija. 
	            
								\begin{center}
										\texttt{cd backend}
								\end{center}
								
								Sada je potrebno instalirati potrebne pakete za uspješan rad aplikacije. To činimo pokretanjem naredbe
								
								\begin{center}
										\texttt{npm install}
								\end{center}
								
								Nakon instalacije paketa, završili smo lokalnu instalaciju servera te server možemo pokrenuti naredbom
								
								\begin{center}
										\texttt{npm run dev}
								\end{center}
								
								Sada je server pokrenut te mu možemo pristupiti koristeći naredbeni redak i naredbu \texttt{curl} ili pak neki drugi alat za kreiranje HTTP zahtjeve poput alata \underline{Postman} \footnote{\href{https://www.postman.com/}{Postman}}.
		             
						\pagebreak
		    
						\subsubsection*{Heroku instalacija}
	        
								Kako bismo backend aplikaciju pustili u pogon, potrebno je kreirati novu aplikaciji na Heroku platformi.
								
								\begin{enumerate}
									\item Otvoriti \href{https://dashboard.heroku.com/apps}{heroku.com} stranicu i prijaviti se ukoliko to već nismo prethodno učinili.
									\item U gornjem desnom kutu možemo vidjeti gumb \textit{New} te kada ga pritisnemo otvara nam se padajuća lista i u njoj odabiremo \textit{Create new app}
									\item Otvara nam se forma za kreiranje nove aplikacije te tu navodimo ime aplikacije koje može biti proizvoljno. Također na tom koraku odabiremo i regiju servera i tu je poželjno postaviti server koji je nama najbliži, a to je Europa.
									\item Pritiskom na \textit{Create app} stvaramo aplikaciju
		        		\end{enumerate}
		        
		        		Sada kada je Heroku aplikacija napravljena potrebno je otvoriti \texttt{.git/config} datoteku unutar korijenskog direktorija aplikacije te na kraj te datoteke potrebno je dodati

                \begin{verbatim}
    [remote "heroku-backend"]
        	url = https://git.heroku.com/${imeHerokuAplikacije}.git
        	fetch = +refs/heads/*:refs/remotes/heroku-backend/*
								\end{verbatim}
								
								te umjesto \texttt{\{imeHerokuAplikacije\}} potrebno je umetnuti ime novostvorene aplikacije. Kada smo podesili \texttt{config} datoteku sve što je potrebno napraviti je pozicionirati se u direktorij backend aplikacije te iz terminala pokrenuti 

								\begin{center}
										\texttt{npm run deploy}
								\end{center}
								
								Nakon par minuta aplikacija će biti puštena u pogon.
						
						\pagebreak
		    
				\subsection{Baza podataka}
			
						\subsubsection*{Lokalna instalacija}
						
								Potrebno je preuzeti \textit{pgAdmin} aplikaciju uz koju standardno dolazi i \textit{PostgreSQL} server. Pokrenuti instalaciju te uz \textit{pgAdmin} klijenta odabrati i \textit{PostgreSQL} server za instalaciju. Nakon instalacije potrebno je postaviti korisnika. Nakon instalacije \textit{pgAdmin} klijent i sam \textit{PostgreSQL} server su pokrenuti.
								
						\subsubsection*{Lokalna konfiguracija}
								
								Nakon instalacije, potrebno je napraviti novu lokalnu bazu.
								
								\begin{enumerate}
										\item Proširiti \textit{Servers} te, ovisno o preuzetoj verziji \textit{PostgreSQL} servera, proširiti  \textit{PostgreSQL XX}
										\item Desni klik na \textit{Databases} te \textit{Create} te \textit{Database...}
										\item Otvara se modal te kao \textit{Database} vrijednost stavljamo proizvoljni naziv
										\item Odabiremo \textit{Spremi} te time stvaramo novu bazu
								\end{enumerate}
								
								Nakon stvorene baze potrebno je stvoriti konekcijski \textit{connection string}. Njega ćemo stvoriti što ćemo vrijednosti u vitičastim zagradama zamijeniti s pravim vrijednostima.
								
								\begin{center}
										\texttt{postgres://{username}:{password}@{hostname}:{port}/{database}}
								\end{center}
								
								\begin{enumerate}
										\item username - inicijalni username je \textit{postgres}
										\item password - lozinka postavljena prilikom instalacije
										\item hostname - pošto je server lokalni, hostname će biti \textit{localhost}
										\item port - inicijalni priključak PostgreSQL servera je \textit{5432}
										\item database - ime novo-napravljene baze
								\end{enumerate}
								
								Nakon generiranog \textit{connection stringa}, kreirat ćemo \textit{.env} datoteku u korijenskom direktoriju backend aplikacije te u njoj napraviti zapis:

								\begin{center}
										\texttt{DATABASE\_URL={connectionString}}
								\end{center}
								
								Sada će prilikom pokretanja backend aplikacije, prije samog spajanja s bazom, aplikacija pročitati sadržaj \textit{.env} datoteke i učitati te vrijednosti kao sistemske varijable. U \textit{Node.js} aplikaciji tim varijablama imamo pristup preko

								\begin{center}
										\texttt{proccess.env.\{imeVarijable\}}
								\end{center}
				
				
						\subsubsection*{Heroku instalacija}
						
								\begin{enumerate}
										\item Otići na Heroku Dashboard
										\item Odabrati kreiranu backend aplikaciju
										\item Odabrati karticu \textit{Resources}
										\item Unutar \textit{Add-ons} tražilice tražiti \textbf{Heroku Postgres}
										\item Kada se otvori modal, odabrati \textbf{Submit Order Form}
								\end{enumerate}
								
								I to je to! Server sada ima svoju bazu. Ukoliko se želimo spojiti na tu bazu preko pgAdmin klijenta tada slijedimo sljedeće korake:
								
								\begin{enumerate}
										\item S \textit{Resources} zaslona odaberemo naš \textit{Heroku Postgres} server
										\item Nakon malo duljeg učitavanja, odabiremo karticu \textit{Settings}
										\item Odabiremo gumb \textit{View Credentials...}
								\end{enumerate}
								
								Sada na su nam na ekranu prikazani svi potrebni podaci za spajanje na PostgreSQL server. Sljedeći korak je otići u pgAdmin klijent te kreirati novu konekciju.
								
								\begin{enumerate}
										\item Otvoriti pgAdmin
										\item Desnim klikom kliknuti na \textit{Servers} te odabrati \textit{Create} \textit{Server...}
										\item Postaviti proizvoljno ime (ne utječe na samu konekciju)
										\item Odabrati karticu \textit{Connection} te tu popuniti potrebne podatke (\textit{Hostname/address}, \textit{Maintenece database}, \textit{Username}, \textit{Password})
										\item Nakon popunjenih podataka potrebno je kliknuti na gumb \textit{Save}
								\end{enumerate}
								
								Sada smo stvorili novu konekciju te iz popisa servera možemo odabrati novo-kreirani te manipulirati s tablicama.
								
				
						\subsubsection*{Heroku konfiguracija}
							
							Slično kao u lokalnoj konfiguraciji potrebno je igrati se s varijablama. Jedina i vrlo bitna razlika je ta da sve već posloženo te na Heroku serveru već postoji varijabla \textit{DATABASE\_URL}. Potrebno je samo pustiti backend aplikaciju u pogon.
					
			\eject